\documentclass{article}
\usepackage[utf8]{inputenc}

\title{\vspace{-3cm}Philosophy of Artificial Intelligence}
\author{Neelkant Newra}
\date{July 2021}
\begin{document}
\maketitle
Artificial intelligence philosophy is a subset of technology philosophy that investigates artificial intelligence and its implications for knowledge and understanding of intelligence, ethics, consciousness, epistemology, and free will. Furthermore, because the technology is involved with the construction of artificial animals or humans (or, at the very least, artificial beings; see artificial life), philosophers are interested in the field. These elements all played a role in the development of artificial intelligence philosophy. Some academics believe that the AI community's dismissive attitude toward philosophy is harmful.\\

The philosophy of artificial intelligence attempts to answer such questions as follows:
\begin{itemize}
	\item Is it possible for a machine to react intelligently? Is it capable of solving every problem that a person might solve by reasoning?

	\item Is there a difference between human and computer intelligence? Is the human brain a computer in disguise?

	\item Is it possible for a computer to have a mind, mental states, and awareness in the same way that a person does? Is it able to sense how things are?
	
	\item Can a machine reach the Human level performance?

\end{itemize}

Is it feasible to build a machine that can solve all of the issues that people solve with intelligence? This question determines the extent of what machines may be able to accomplish in the future, and neurology directs AI research. "Every component of learning or any other attribute of intelligence may be so clearly defined that a computer can be created to imitate it," said a Dartmouth workshop proposal in 1956, summarising the core stance of most AI researchers.
To address the question, the first step is to define "intelligence." The difficulty of defining intelligence was simplified by Alan Turing to a simple conversational inquiry.He claims that if a machine can respond to every question posed to it in the same language as a human would, then we may label it intelligent. Turing's test applies this polite norm to machines: if a computer acts intelligently, it is intelligent.\\
This is a philosophical topic that is linked to the problem of other minds as well as the difficult problem of consciousness. The debate centres on John Searle's "strong AI" stance, which says that "a physical symbol system can have a mind and mental states." This is distinct from what Searle referred to as "weak AI," in which a physical symbol system may behave intelligently. Searle used the words to distinguish between strong and weak AI, allowing him to focus on what he considered to be the more fascinating and disputed topic. Even if we assumed we had a computer programme that behaved precisely like a human mind, he said, there would still be a tough philosophical matter to resolve.\\

Artificial Intelligence encompasses a wide range of topics:

\begin{enumerate}
	\item \textbf{Intelligent Agent: }An intelligent agent is a software program that can make decisions or provide services depending on its surroundings, human input, and previous experiences. These applications can be used to gather data on a regular, pre-programmed schedule or when the user prompts them in real time.

	\item \textbf{Problem Solving: }It is a part of artificial intelligence that encompasses a number of techniques such as a tree, heuristic algorithms to solve a problem. We can also say that a problem solving agent is a result driven agent and always focuses on satisfying the goals.

	\item \textbf{Knowledge}: It is the information about a domain that can be used to solve problems in that domain. As part of designing a program to solve problems, we must define how the knowledge will be represented. A representation scheme is the form of the knowledge that is used in an agent.

	\item \textbf{Reasoning: }The reasoning is the mental process of deriving logical conclusion and making predictions from available knowledge, facts, and beliefs.

	\item \textbf{Planning:} It is about the decision$-$making tasks performed by the robots or computer programs to achieve a specific goal. The execution of planning is about choosing a sequence of actions with a high likelihood to complete the specific task.

	\item \textbf{Uncertain knowledge}: When the available knowledge has multiple causes leading to multiple effects or incomplete knowledge of causality in the domain.

	\item \textbf{Machine Learning: }It is a process that improves the knowledge of an AI program by making observations about its environment.

	\item \textbf{Communicating:} Natural language processing includes technology like machine translation of human languages, spoken conversation systems like Siri, algorithms capable of creating publishable journalistic material, and social robots, all of which are meant to connect with people in a human$-$like manner.'

	\item \textbf{Perception:} In Artificial Intelligence it is the process of interpreting vision, sounds, smell, and touch. Perception is a process to interpret, acquire, select, and then organize the sensory information from the physical world to make actions like humans.

	\item \textbf{Acting:} It refers to the action done by the algorithm in real world based on the processed data the algorithm has been fed.\end{enumerate}
	
\end{document}



\end{document}
