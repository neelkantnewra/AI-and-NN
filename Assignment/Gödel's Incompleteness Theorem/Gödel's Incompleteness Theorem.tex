\documentclass{article}
\usepackage[utf8]{inputenc}

\title{\vspace{-2cm}Gödel's Incompleteness Theorem}
\author{Neelkant Newra }
\date{July 2021}

\begin{document}

\maketitle 
\paragraph{Gödel's incompleteness theorems are two theorems of mathematical logic that are concerned with the limits of provability in formal axiomatic theories. According to the first incompleteness theorem, no consistent set of axioms whose theorems can be stated by an effective process (i.e., an algorithm) can prove all facts about natural number arithmetic.}

\paragraph{Formal systems: completeness, consistency, and effective axiomatization - The incompleteness theorems apply to formal systems that are of sufficient complexity to express the basic arithmetic of the natural numbers and which are consistent, and effectively axiomatized, these concepts being detailed below.Completeness: A set of axioms is (syntactically, or negation-) complete if, for any statement in the axioms' language, that statement or its negation is provable from the axioms.}

\paragraph{In Gödel's 1931 work "On Formally Undecidable Propositions of Principia Mathematica and Related Systems I," the first incompleteness theorem appeared as "Theorem VI."
The theorem's unprovable assertion GF is commonly referred to as "the Gödel sentence" for the system F. The proof creates a specific Gödel sentence for the system F, although there are an unlimited number of statements in the system's language that possess the same features, such as the conjunction of the Gödel sentence with any logically acceptable phrase.}

\paragraph{The second incompleteness theorem is based on the fact that there are two types of incomplete It is feasible to canonically define a formula Cons(F) that expresses the consistency of any formal system F that contains fundamental arithmetic. "There does not exist a natural number coding a formal derivation within the system F whose conclusion is a syntactic contradiction," says this formula."}
\end{document}
